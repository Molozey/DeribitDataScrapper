\documentclass[11pt, oneside]{article}   	% use "amsart" instead of "article" for AMSLaTeX format
\usepackage{geometry}                		% See geometry.pdf to learn the layout options. There are lots.
\geometry{letterpaper}                   		% ... or a4paper or a5paper or ... 
%\geometry{landscape}                		% Activate for rotated page geometry
%\usepackage[parfill]{parskip}    		% Activate to begin paragraphs with an empty line rather than an indent
\usepackage{graphicx}
\usepackage[russian]{babel}				% Use pdf, png, jpg, or eps§ with pdflatex; use eps in DVI mode
\usepackage[utf8]{inputenc}
\usepackage{amsmath}					% TeX will automatically convert eps --> pdf in pdflatex

%SetFonts

%SetFonts


\title{Option Market Data Scrapper}
\author{Andrii Lipskii}
%\date{}							% Activate to display a given date or no date

%! suppress = Unicode
\begin{document}
\maketitle
\section{Scrapper.py}
    Основными функциями являются - call\_api а также send\_request.
    Первая открывает соединение с Deribit передает запрос, вторая - дожидается ответа от сервера.
    Список доступных запросов представлен с файле available\_request.py.
    Запросы могут требовать передачи дополнительных параметров в качестве аргументов функции.
    \subsection{Available Instruments}
        Список доступных инструментов представлен в файле AvailableInstruments.py.
        Инстурмент является чем-то уникальным, как пример опцион на BTC с экспирацией в момент X.
        Получить актуальный список достпных инструментов можно с помощью запроса get\_instruments\_by\_currency\_request.
    \subsection{Available Currencies}
        Список доступных валют, используется для того чтобы получить списки доступных инструментов.
    \subsection{Available Instrument Type}\label{subsec:insrument_type}
        Список достпных типов инструментов, к примеру, может быть опцион или фьючерс.


\section{Receipt of getting Option Surface}
    Сначала необходимо собрать запрос get\_instruments\_by\_currency\_request, передав в качестве параметров валюту для которой мы строим поверхность, а также тип инструмента опцион (\ref{subsec:insrument_type}).


\end{document}  